\documentclass[12pt,a4paper,oneside]{article}
\usepackage[a4paper]{geometry}
\usepackage[utf8]{inputenc}
\usepackage[danish]{babel}
\usepackage[T1]{fontenc}
\usepackage{mathtools}
\usepackage{ulem}
\usepackage{url}
\usepackage{mathpazo}
\usepackage{graphicx}
\usepackage{enumitem}
\usepackage{lastpage}
\usepackage{fancyhdr}
\setlength{\headheight}{45pt}
\pagestyle{fancy}

\renewlist{description}{description}{5}
\setlist[description]{labelindent=3cm,labelwidth=3cm,itemindent=3cm,
leftmargin=3cm}

\lhead{Københavns Universitet \\ DIKU \\ \today}
\rhead{Mikkel Oscar Lyderik Larsen \\ Claus Skou Nielsen \\ Nicolai Willems}
\cfoot{\thepage\ of \pageref{LastPage}}

\begin{document}

\section{Indledning til projektet}
Projektet omhandler implementeringen af en indlånskontotype, kaldet Aktiekonto.

Det går ud på at kunden indbetaler et minimum på ca. 25000 kr. til en konto, og
over tre år får udbyttet af nogle investeringer som banken står for.

Det IT-mæssige ligger i at der er noget teknisk ved at oprette denne type konto,
både fordi der er en pensionsmæssig del som skal behandles på en bestemt måde, og
samtidig skal der oplyses nogle information til skat. Dette er alle ting som man
kan gøre manuelt, men kan effektiviseres meget ved at implementere et system der
kan gøre det. Både for at spare mandetimer, men også for at eliminere chancen
for fejlindtastninger.

Vores kontakt var projektlederen for IT-afdelingen, der står for
implementeringen af projektet. Projektet laves internt ved hjælp af firmaets
egne ressourcer. Strukturen med hensyn til sådanne projekter, svarer
ikke direkte til kunde-, leverandør- og systemejermodellen i Nykredit, men de
har derimod deres egen ordning. Den vil vi ikke komme nærmere ind på i denne
rapport, da det ikke er en del af opgaveformuleringen.

\section{TCO}
TCO står for Total Cost of Ownership, og begrebet dækker ifølge
Marchewka\footnote{Marchewka, Jack T.: Information Technology Project
Management, 3rd edition, Wiley} over de totale omkostninger ved at eje et
projekt. Generelt for IT-projekter, vil det dække over de åbenlyse omkostninger,
såsom konsulentlønninger, hardware, software og salærer betalt til leverandøren.
Det dækker også over indirekte omkostninger, hvis man for eksempel som kunde
skal stille testpersonale til rådighed, eller selv bruge ressourcer på
projektet.

I tilfældet for vores case, vidste vi ikke meget om projektet før vi
interviewede vores kontakt. Vi vidste at det omhandlede strukturerede indlån, og
at IT-projektet dækkede over noget logik i forbindelse med dette. Med denne
sparsomme information in mente, forestillede vi os følgende TCO:

\begin{description}
  \item[Udviklere]
  Omkostninger ved udvikling af systemet. Lønninger etc.
  \item[Planlægning]
  Konkrete udgifter og ressourcer der er blevet brugt på planlægning af systemet
  \item[Test]
  Personale og mandetimer brugt på test af systemet
  \item[Udrulning]
  Ressourcer og mandskab brugt på udrulning af systemet
\end{description}

Da vi havde foretaget vores interview, fik vi lidt mere at vide i forhold til
hvor omkostningerne lå, selv om vi ikke kunne få nogen konkrete tal.

Udvikling af systemet havde kostet noget. Planlægning og analysearbejde havde
ikke taget langt tid. Han nævnte ikke test af systemet, men da de rullede
systemet ud, var der noget data som var uren. De havde forudset at valideringen
af data måske kunne være fejlramt, men ikke i så høj grad som det var tilfældet.
Derfor endte de med at bruge mange flere penge på driften af systemet, end de
havde regnet med, selv om de faktisk havde overvejet risikoen for uren data.

Hardware og software havde slet ikke været en omkostning, da programmellet ikke
skulle køres non-stop, men derimod kun få gange om året, og at de kunne bruge
eksisterende softwarelicenser og hardware til at udvikle og ibrugtage systemet

Generelt virkede det ikke som om der var gjort konkrete overvejelser om, fra
Nykredits side, præcis hvor meget det projektet skulle komme til at koste, andet
end et slag på tasken.  Dette kan skylde projektets størrelse. Vores kontakt
fortalte, at det var et meget lille projekt i Nykredit, hvilket kunne være
grunden til at de ikke havde lavet en så præcis estimering.


\section{TBO}
TBO(Total Benefit of Ownership) er et begreb der beskriver hvor stor gevinsten
ved et projekt er. De generelle forbedringer/gevinster indenfor IT-Projekter
forholder sig til effektiviseringer, øget kundetilfredshed og strategisk
positionering.

I vores case, forventer vi at finde følgende elementer i TBO'en:

\begin{description}
\item[Effektivisering]
\item[Øget indtjening]
\end{description}


\section{MOV}
I business casen defineres en eller flere Målbare Organisatoriske Værdier (MOV).
Disse skal fungere som successkriterier for projektet, og beskæftiger sig
ofte med hvor meget projektet skal effektivisere, optimere eller forbedre et
givent forretningsområde. De er ofte defineret i forhold til virksomhedens
overordnede strategi.

I vores case forestiller vi os at MOVen ligner et eller flere af nedenstående
udsagn:

\begin{itemize}
  \item At forbedre omsætningen ved indlån og pensions opsparinger med N \%

  \item At effektivisere arbejdsgangene omkring struktureret indlån, og gøre
medarbejderen i stand til at behandle N \% flere konti pr. time

  \item At tilfredsstille et kuneønske, M kunder har ønsket produktet, målet er at
M+10\% benytter det nye produkt.
\end{itemize}

Efter vores møde med vores kontakt i virksomheden, har det vist sig at projektet er til
for at øge virksomhedens indtjening, dog med det i mente at produktet som
projektet understøtter sagtens kan køre uden, men at dette ville medføre
massive udgifter i medarbejdertimer.

Der har ikke umiddelbart været noget kundeønske involveret. Hvilket også leder
naturligt i retning af, at det er et produkt der er drevet af virksomhedsstrategi
og øget indtjening.


\end{document}
