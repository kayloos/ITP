\documentclass[12pt,a4paper,oneside]{article}
\usepackage[a4paper]{geometry}
\usepackage[utf8]{inputenc}
\usepackage[danish]{babel}
\usepackage[T1]{fontenc}
\usepackage{mathtools}
\usepackage{ulem}
\usepackage{url}
\usepackage{mathpazo}
\usepackage{fancyhdr}
\usepackage{graphicx}
\usepackage{enumerate}
\usepackage{lastpage}
\usepackage{fancyhdr}
\setlength{\headheight}{45pt}
\pagestyle{fancy}

\lhead{Københavns Universitet \\ DIKU \\ \today}
\rhead{Mikkel Oscar Lyderik Larsen \\ Claus Skou Nielsen \\ Nicolai Willems}
\cfoot{\thepage\ of \pageref{LastPage}}

\begin{document}
\section{Indledning til projektet}
Projektet omhandler implementeringen af en indlånskontotype, kaldet Aktiekonto.

Det går ud på at kunden indbetaler et minimum på ca. 25000 kr. til en konto, og
over 3 år får udbyttet af nogle invisteringer som banken står får.

Det IT-mæssige ligger i at der er noget teknisk ved at oprette denne type konto,
både fordi der er en pensionsmæssig del som skal behandles på en bestemt måde, og
samtidig skal der oplyses nogle information til skat. Dette er alle ting som man
kan gøre manuelt, men kan effektiviseres meget ved at implementere et system der
kan gøre det. Både for at spare mandetimer, men også for at eleminerer chancen
for fejlindtastninger.
  \section{TCO}
  TCO står for Total Cost of Ownership, og begrebet dækker over de totale
omkostninger ved at eje et projekt. Generelt for IT-projekter, vil det dække
over de åbenlyse omkostninger, såsom konsulentlønninger, hardware, software og
salærer betalt til leverandøren. Det dækker også over indirekte omkostninger,
hvis man for eksempel som kunde skal stille testpersonale til rådighed, eller
selv bruge ressourcer på projektet.

I vores case, har vi en forventning om at følgende elementer indgår i TCO'en:

\begin{description}
\item[Udviklere]
\item[Planlægning]
\item[Test]
\item[Udrulning]
\end{description}


  \section{TBO}
  TBO (Total Benefit of Ownership) er et begreb der beskriver hvor stor gevinsten
ved et projekt er. De generelle forbedringer/gevinster indenfor IT-Projekter
forholder sig til effektiviseringer, øget kundetilfredshed og strategisk
positionering.

I vores case, forventer vi at finde følgende elementer i TBO'en:

\begin{itemize}
  \item Effektivisering
  \item Øget indtjening
\end{itemize}

Der er to dele af projektet ved vores case. Det ene er projektet som helhed, og
det andet er selve IT-delen i projektet. Ved projektet som helhed er TBO helt
tydeligt en øget indtjening. Målet er at tilbyde et nyt produkt som virksomheden
kan tjene penge på. Eftersom kontoerne står i 3 år er det dog først muligt at
se hvad den faktiske indtjening er efter denne periode, og er dermed lidt svært
at estimere på nuværende tidspunkt.

Derudover kan man også forvente at der er nogle kunder som bliver meget glade
for denne nye kontotype og vælger denne virksomhed frem for en anden. Det kan
også tænkes at kunder som benytter sig af denne type konto også vælger andre
typer af kontoer i virksomheden. En måde hvorpå man kunne måle gevinsten i
forhold til nye kunder, kunne gøres ved brug af et spørgeskema.

Ved IT-delen af projektet er TBO at få effektiviseret nogle processer som
alternativt skulle være håndteret manuelt. Dette må forventes at give et
overskud i forhold til at firmaet skal bruge færre midler på mandetimer. Ved
manuelt arbejde er der også risiko for fejl, som mere eller mindre kan regnes
over i mandetimer når disse skal rettes. Dette giver naturligvis også en
besparelse.


  \section{MOV}
  I business casen defineres en eller flere Målbare Organisatoriske Værdier(MOV),
disse skal fungerer som success kriterier for projektet, og beskæftiger sig
ofte med hvor meget projektet skal effektiviserer, optimere eller forbedre et
givent forretnings område. De er ofte defineret i forhold til virksomhedens
overordnede strategi.

I vores case forstiller vi os at MOVen ligner en eller flere af nedenstående
udsagn:

\begin{quote}
At forbedre omsætningen ved indlån og pensions opsparinger med N \%

At effektiviserer arbejdsgangene omkring struktureret indlån, og gøre
medarbejderen i stand til at behandle N \% flere konti pr. time

At tilfredsstille et kuneønske, M kunder har ønsket produktet, målet er at
M+10\% benytter det nye produkt.
\end{quote}

Efter vores møde med PLen i virksomheden, har det vist sig at projektet er til
for at øge virksomhedens indtjening, dog med den mente at produktet som
projektet understøtter sagtens kan kører uden, men at dette ville medføre
massive udgifter i medarbejder timer.

Der har ikke umiddelbart været noget kundeønske involveret. Hvilket også leder
naturligt i retning af at det er et produkt der er drevet af virksomhedsstrategi
og øget indtjening.

\end{document}
