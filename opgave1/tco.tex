TCO står for Total Cost of Ownership, og begrebet dækker ifølge
Marchewka\footnote{Marchewka, Jack T.: Information Technology Project
Management, 3rd edition, Wiley} over de totale omkostninger ved at eje et
projekt. Generelt for IT-projekter, vil det dække over de åbenlyse omkostninger,
såsom konsulentlønninger, hardware, software og salærer betalt til leverandøren.
Det dækker også over indirekte omkostninger, hvis man for eksempel som kunde
skal stille testpersonale til rådighed, eller selv bruge ressourcer på
projektet.

I tilfældet for vores case, vidste vi ikke meget om projektet før vi
interviewede vores kontakt. Vi vidste at det omhandlede strukturerede indlån, og
at IT-projektet dækkede over noget logik i forbindelse med dette. Med denne
sparsomme information in mente, forestillede vi os følgende TCO:

\begin{description}
  \item[Udviklere]
  Omkostninger ved udvikling af systemet. Lønninger etc.
  \item[Planlægning]
  Konkrete udgifter og ressourcer der er blevet brugt på planlægning af systemet
  \item[Test]
  Personale og mandetimer brugt på test af systemet
  \item[Udrulning]
  Ressourcer og mandskab brugt på udrulning af systemet
\end{description}

Da vi havde foretaget vores interview, fik vi lidt mere at vide i forhold til
hvor omkostningerne lå, selv om vi ikke kunne få nogen konkrete tal.

Udvikling af systemet havde kostet noget. Planlægning og analysearbejde havde
ikke taget langt tid. Han nævnte ikke test af systemet, men da de rullede
systemet ud, var der noget data som var uren. De havde forudset at valideringen
af data måske kunne være fejlramt, men ikke i så høj grad som det var tilfældet.
Derfor endte de med at bruge mange flere penge på driften af systemet, end de
havde regnet med, selv om de faktisk havde overvejet risikoen for uren data.

Hardware og software havde slet ikke været en omkostning, da programmellet ikke
skulle køres non-stop, men derimod kun få gange om året, og at de kunne bruge
eksisterende softwarelicenser og hardware til at udvikle og ibrugtage systemet

Generelt virkede det ikke som om der var gjort konkrete overvejelser om, fra
Nykredits side, præcis hvor meget det projektet skulle komme til at koste, andet
end et slag på tasken.  Dette kan skylde projektets størrelse. Vores kontakt
fortalte, at det var et meget lille projekt i Nykredit, hvilket kunne være
grunden til at de ikke havde lavet en så præcis estimering.
