TCO står for Total Cost of Ownership, og begrebet dækker over de totale
omkostninger ved at eje et projekt. Generelt for IT-projekter, vil det dække
over de åbenlyse omkostninger, såsom konsulentlønninger, hardware, software og
salærer betalt til leverandøren. Det dækker også over indirekte omkostninger,
hvis man for eksempel som kunde skal stille testpersonale til rådighed, eller
selv bruge ressourcer på projektet.

I vores case, har vi en forventning om at følgende elementer indgår i TCO'en:

\begin{description}
\item[Udviklere]
\item[Planlægning]
\item[Test]
\item[Udrulning]
\end{description}
