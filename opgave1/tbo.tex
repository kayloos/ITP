TBO (Total Benefit of Ownership) er et begreb der beskriver hvor stor gevinsten
ved et projekt er. De generelle forbedringer/gevinster indenfor IT-Projekter
forholder sig til effektiviseringer, øget kundetilfredshed og strategisk
positionering.

I vores case, forventer vi at finde følgende elementer i TBO'en:

\begin{description}
\item[Effektivisering]
\item[Øget indtjening]
\end{description}

Der er to dele af projektet ved vores case. Det ene er projektet som helhed, og
det andet er selve IT-delen i projektet. Ved projektet som helhed er TBO helt
tydeligt en øget indtjening. Målet er at tilbyde et nyt produkt som virksomheden
kan tjene penge på. Eftersom kontoerne står i 3 år er det dog først muligt at
se hvad den faktiske indtjening er efter denne periode, og er dermed lidt svært
at estimere på nuværende tidspunkt. Derudover kan man også forvente at der er
nogle kunder som bliver meget glade for denne nye kontotype og vælger denne
virksomhed frem for en anden. Det kan også tænkes at kunder som benytter sig af
denne type konto også vælger andre typer af kontoer i virksomheden. En måde
hvorpå man kunne måle gevinsten i forhold til nye kunder, kunne gøres ved brug
af et spørgeskema.

Ved IT-delen af projektet er TBO at få effektiviseret nogle processer som
alternativt skulle være håndteret manuelt. Dette må forventes at give et
overskud i forhold til at firmaet skal bruge færer midler på mandetimer. Ved
manuelt arbejder er der også risiko for fejl, som mere eller mindre kan regnes
over i mandetimer når disse skal rettes. Dette giver naturligvis også en
besparelse.
