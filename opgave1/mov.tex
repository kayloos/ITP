I business casen defineres en eller flere Målbare Organisatoriske Værdier (MOV).
Disse skal fungere som successkriterier for projektet, og beskæftiger sig
ofte med hvor meget projektet skal effektivisere, optimere eller forbedre et
givent forretningsområde. De er ofte defineret i forhold til virksomhedens
overordnede strategi.

I vores case forestiller vi os at MOVen ligner et eller flere af nedenstående
udsagn:

\begin{itemize}
  \item At forbedre omsætningen ved indlån og pensions opsparinger med N \%

  \item At effektivisere arbejdsgangene omkring struktureret indlån, og gøre
medarbejderen i stand til at behandle N \% flere konti pr. time

  \item At tilfredsstille et kuneønske, M kunder har ønsket produktet, målet er at
M+10\% benytter det nye produkt.
\end{itemize}

Efter vores møde med vores kontakt i virksomheden, har det vist sig at projektet er til
for at øge virksomhedens indtjening, dog med det i mente at produktet som
projektet understøtter sagtens kan køre uden, men at dette ville medføre
massive udgifter i medarbejdertimer.

Der har ikke umiddelbart været noget kundeønske involveret. Hvilket også leder
naturligt i retning af, at det er et produkt der er drevet af virksomhedsstrategi
og øget indtjening.
