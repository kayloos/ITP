I business casen defineres en eller flere Målbare Organisatoriske Værdier(MOV),
disse skal fungerer som success kriterier for projektet, og beskæftiger sig
ofte med hvor meget projektet skal effektiviserer, optimere eller forbedre et
givent forretnings område. De er ofte defineret i forhold til virksomhedens
overordnede strategi.

I vores case forstiller vi os at MOVen ligner en eller flere af nedenstående
udsagn:

\begin{quote}
At forbedre omsætningen ved indlån og pensions opsparinger med N \%

At effektiviserer arbejdsgangene omkring struktureret indlån, og gøre
medarbejderen i stand til at behandle N \% flere konti pr. time

At tilfredsstille et kuneønske, M kunder har ønsket produktet, målet er at
M+10\% benytter det nye produkt.
\end{quote}

Efter vores møde med PLen i virksomheden, har det vist sig at projektet er til
for at øge virksomhedens indtjening, dog med den mente at produktet som
projektet understøtter sagtens kan kører uden, men at dette ville medføre
massive udgifter i medarbejder timer.

Der har ikke umiddelbart været noget kundeønske involveret. Hvilket også leder
naturligt i retning af at det er et produkt der er drevet af virksomhedsstrategi
og øget indtjening.
